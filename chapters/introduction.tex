\chapter {Giới thiệu}\label{chapter_intro}
	
\section{Đặt vấn đề}
Đồ án/khóa luận tốt nghiệp đại học là một công trình hoàn chỉnh sinh viên nhằm tổng hợp, vận dụng kiến thức đã học để trình bày phát hiện mới hoặc thiết kế, xây dựng một hệ thống/sản phẩm. Thông thường, sinh viên thực hiện đồ án tốt nghiệp dưới sự hướng dẫn của một hoặc một số giảng viên có kinh nghiệm trong lĩnh vực nghiên cứu.

Tài liệu này giới thiệu một khuôn mẫu (template) giúp sinh viên soạn thảo đồ án tốt nghiệp một cách chuyên nghiệp với \LaTeX \cite{van2010documentation}.

\subsection{Ứng dụng Học máy xây dựng hệ thống nhận dạng hình ảnh}
Nhập nội dung của subsection ở đây.

\section{Mục tiêu của đề tài}

Các mục tiêu chính của đề tài bao gồm:
\begin{itemize}
\item Tìm hiểu tổng quan về Trí tuệ nhân tạo và ứng dụng;
\item Tìm hiểu cơ sở toán học và mô hình học máy;
\item Xây dựng hệ thống nhận dạng loài vật tự động qua ảnh, sử dụng mô hình học máy tiên tiến;
\item Triển khai mô hình thành ứng dụng.
\end{itemize}

\section{Cấu trúc của Đồ án}
Đồ án gồm các phần như sau:
\begin{itemize}
\item Chương \ref{chapter_intro}: Giới thiệu. 
\item Chương \ref{chapter_overview}: Tổng quan.
\item Chương \ref{chapter_conclusion}: Kết luận.
\end{itemize}
